Falls Sie eigene Beispiele testen möchten,
rufen Sie das Programm bitte folgendermaßen auf:

\shellcmd{./zimmerbelegung.x dateiname}

Getestet habe ich meinen Quellcode unter gcc 7.1.1-3 auf einem 64bit-Fedora.
Mit folgendem Kommando können Sie mein Programm selbst kompilieren:

\shellcmd{g++ -O -o zimmerbelegung.x main.cpp girl.cpp}

Hier finden Sie die Ausgaben zu allen Beispielen der BwInf-Website:

\lstinputlisting[breaklines=true]{../Beispiele/Ausgaben/z1.txt}
\lstinputlisting[breaklines=true]{../Beispiele/Ausgaben/z2.txt}
\lstinputlisting[breaklines=true]{../Beispiele/Ausgaben/z3.txt}
\lstinputlisting[breaklines=true]{../Beispiele/Ausgaben/z4.txt}
\lstinputlisting[breaklines=true]{../Beispiele/Ausgaben/z5.txt}
\lstinputlisting[breaklines=true]{../Beispiele/Ausgaben/z6.txt}

Zur weiteren Überprüfung meines Programmes habe ich einige weitere Testfälle konstruiert.

In \texttt{keineWuensche.txt} ist allen Mädchen die Zimmerbelegung völlig egal,
sie haben weder Wünsche noch Konflikte.
Folglich werden alle in Einzelzimmern platziert.
\lstinputlisting[breaklines=true]{../Beispiele/Ausgaben/keineWuensche.txt}

In \texttt{everybodysDarling.txt} möchten alle mit "`Luisa"' wohnen,
diese hat weder Wünsche noch Konflikte.
Folglich werden alle in einem gemeinsamen Zimmer platziert.
\lstinputlisting[breaklines=true]{../Beispiele/Ausgaben/everybodysDarling.txt}

In \texttt{hassliebe.txt} existiert ein Widerspruch, "`Hanna"' gibt "`Franziska"'
gleichzeitig als Wunsch als auch als Konflikt an.
Folglich wird korrekterweise "`Impossible"' ausgegeben.
\lstinputlisting[breaklines=true]{../Beispiele/Ausgaben/hassliebe.txt}
