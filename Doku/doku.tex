\documentclass[a4paper,10pt,ngerman]{scrartcl}
\usepackage[ngerman]{babel}

\usepackage[T1]{fontenc}
\usepackage[utf8]{inputenc}
\usepackage[a4paper,margin=2.5cm]{geometry}

% Die nächsten drei Felder bitte anpassen:
\newcommand{\Name}{Laurenz Grote} % Teamname oder eigenen Namen angeben
\newcommand{\Einsendenummer}{Team 00905 / Teilnahme 44325}
\newcommand{\Aufgabe}{Aufgabe 1: Zimmerbelegung}

% Aliase zum abdrucken von Shell-CMDs
\newcommand{\shellcmd}[1]{\texttt{\$ #1}\\}
\newcommand{\shellout}[1]{\texttt{#1}\\}

% Kopf- und Fußzeilen
\usepackage{scrlayer-scrpage}
\setkomafont{pageheadfoot}{\textrm}
\ifoot{\Name}
\cfoot{\thepage}
\chead{\Aufgabe}
\ofoot{\Einsendenummer}

% Für mathematische Befehle und Symbole
\usepackage{amsmath}
\usepackage{amsthm}
\usepackage{amssymb}
\newtheorem{annahme}{Annahme}[section]

% Für Bilder
\usepackage{graphicx}

% Für Quelltext
\usepackage{listings}
\usepackage{xcolor}
\definecolor{mygreen}{rgb}{0,0.6,0}
\definecolor{mygray}{rgb}{0.5,0.5,0.5}
\definecolor{mymauve}{rgb}{0.58,0,0.82}
\lstset{language={C++},
  keywordstyle=\color{blue},commentstyle=\color{mygreen},
  stringstyle=\color{mymauve},rulecolor=\color{black},
  basicstyle=\footnotesize\ttfamily,numberstyle=\tiny\color{mygray},
  captionpos=t, % sets the caption-position to bottom
  keepspaces=true, % keeps spaces in text
  numbers=left, numbersep=5pt, showspaces=false,showstringspaces=true,
  showtabs=false, stepnumber=1, tabsize=2, title=\lstname,
  literate=%
    {Ö}{{\"O}}1
    {Ä}{{\"A}}1
    {Ü}{{\"U}}1
    {ß}{{\ss}}1
    {ü}{{\"u}}1
    {ä}{{\"a}}1
    {ö}{{\"o}}1
}
% Für die Bibliographie
\usepackage[babel,german=quotes]{csquotes}
\usepackage[citestyle=authortitle-icomp]{biblatex}
\addbibresource{literatur.bib}

% Diese beiden Pakete müssen als letztes geladen werden
\usepackage{hyperref} % Anklickbare Links im Dokument

% Daten für die Titelseite
\title{\Aufgabe}
\author{\Name / \Einsendenummer}
\date{27. November 2017}

\begin{document}
\maketitle
\tableofcontents

\section{Lösungsidee}
\subsection{Abstraktion}
Die Aufgabenstellung habe ich als Aufgabe aus dem Bereich der Graphentheorie
interpretiert.

Im Kontext der Graphentheorie entsprechen Mädchen Knoten.
Zimmerbelegungswünsche entsprechen folglich Kanten.
Im entstehenden Graphen zeigt dann beispielweise im ersten Beispiel eine Kante von Anna
zu Paula, da Anna sich ein Zimmer mit dieser wünscht.
Nach Einlesen der Belegungswünsche entsteht somit ein gerichteter Graph.

Die möglichen Konflikte, also die Mädchen, mit denen ein Mädchen auf keinen Fall
in einem Zimmer sein möchte, können ebenso in einem gerichteter Graphen verwaltet werden.
Hier entspricht eine Kante dann einer augeschlossenen gemeinsamen Platzierung der beiden Knoten der Kante.

\subsection{Ermittlung der Zimmerbelegung}
\begin{annahme} \label{theo:annahme}
Schwache Zusammenhängigkeitskomponenten im Belegungswunschgraphen entsprechen den Zimmerbelegungen unter Berücksichtigung aller Wünsche.
Schwache Zusammenhangskomponenten\autocite[Definiton. Gerichtete Graphen]{WikiZus}
entsprechen den Teilgraphen des Graphens, in  denen zwischen jedem Knoten eine Verbindung über einen beliebig langen Pfad bestünde,
ersetzte man alle gerichteten Kanten durch entsprechende ungerichtete.
\end{annahme}

Diese Annahme ist korrekt, da sowohl ein einseitiger als auch ein beidseitiger Zimmerwunsch zur Folge hat, dass beide Mädchen sich ein Zimmer teilen müssen. Daher können die gerichten Kanten wie ungerichtete Kanten behandelt werden.

Außerdem muss Mädchen A, welches sich ein Zimmer mit Mädchen B wünscht, auch in einem Zimmer mit Mädchen C, welches sich ebenfalls ein Zimmer mit Mädchen B wünscht, sein. Die Belegungswünsche wirken sich also im Graphen \textit{transitiv} aus. Daher kann der Pfad, über den sich die Notwendigkeit eines gemeinsamen Zimmers ergibt, beliebig lang sein.

\subsection{Überprüfung der Validität der Zimmerbelegung}
Um zu überprüfen, ob die nach Annahme \ref{theo:annahme} berechnete Zimmerverteilung im Sinne der Aufgabenstellung gültig ist, reicht es,
für jede Konfliktkante zu überprüfen, ob beide Knoten der Kante im gleichen Zimmer sind.
Wenn dies für keine Konfliktkante der Fall ist, befinden sich alle Mädchen, zwischen denen ein Konflikt besteht, in verscheidenen Zimmern.

Zur Ausgabe der errechneten Zimmerbelegungen müssen alle Mädchen innerhalb einer Zusammenhangskomponente ausgegeben werden.


\section{Umsetzung}
\subsection{Parsing}

\subsection{Ermittlung der Zusammenhangskomponenten}
Da für die oben aufgeführte Lösungsidee die genaue Struktur des Graphens von keiner Relevanz ist, ist eine traditionelle Abbildung des Graphens mittels einer Adjazenzliste oder einer Adjazenzmatrix nicht nötig.
Da nur die Information der Zugehörigkeit zu einer Zusammenhangskomponente relevant ist, bietet sich eine Union-Find-Datenstruktur\autocite[S. 238]{Sedgewick2014} an. Diese ordnet jedem Knoten eine Zusammenhangskomponente zu. Die genauen Kanten werden hingegen nicht gespeichert.

Da eine typische Klasse eine größe von 30 Mädchen nicht überschreitet, habe ich unter den in der Literaturquelle vorgestellten Implementationen "`Quick-Find"'\autocite[S. 245]{Sedgewick2014} für ausreichend effizient befunden.

In meiner Implementation sind alle Knoten (Mädchen) Schlüssel einer Map. Als Wert haben sie eine Person in deren Zimmer sie sind. Der Wert ist hierbei ein Zeiger auf die Mapposition des jeweiligen Mädchens. Wichtig ist hierbei, dass es für jedes Zimmer das gleiche Mädchen als Identifier dient.



\section{Beispiele}
Falls Sie eigene Beispiele testen möchten,
rufen Sie das Programm bitte folgendermaßen auf:

\shellcmd{./zimmerbelegung.x dateiname}

Getestet habe ich meinen Quellcode unter gcc 7.1.1-3 auf einem 64bit-Fedora.
Mit folgendem Kommando können Sie mein Programm selbst kompilieren:

\shellcmd{g++ -O -o zimmerbelegung.x main.cpp girl.cpp}

Hier finden Sie die Ausgaben zu allen Beispielen der BwInf-Website:

\lstinputlisting[breaklines=true]{../Beispiele/Ausgaben/z1.txt}
\lstinputlisting[breaklines=true]{../Beispiele/Ausgaben/z2.txt}
\lstinputlisting[breaklines=true]{../Beispiele/Ausgaben/z3.txt}
\lstinputlisting[breaklines=true]{../Beispiele/Ausgaben/z4.txt}
\lstinputlisting[breaklines=true]{../Beispiele/Ausgaben/z5.txt}
\lstinputlisting[breaklines=true]{../Beispiele/Ausgaben/z6.txt}

Zur weiteren Überprüfung meines Programmes habe ich einige weitere Testfälle konstruiert.

In \texttt{keineWuensche.txt} ist allen Mädchen die Zimmerbelegung völlig egal,
sie haben weder Wünsche noch Konflikte.
Folglich werden alle in Einzelzimmern platziert.
\lstinputlisting[breaklines=true]{../Beispiele/Ausgaben/keineWuensche.txt}

In \texttt{everybodysDarling.txt} möchten alle mit "`Luisa"' wohnen,
diese hat weder Wünsche noch Konflikte.
Folglich werden alle in einem gemeinsamen Zimmer platziert.
\lstinputlisting[breaklines=true]{../Beispiele/Ausgaben/everybodysDarling.txt}

In \texttt{hassliebe.txt} existiert ein Widerspruch, "`Hanna"' gibt "`Franziska"'
gleichzeitig als Wunsch als auch als Konflikt an.
Folglich wird korrekterweise "`Impossible"' ausgegeben.
\lstinputlisting[breaklines=true]{../Beispiele/Ausgaben/hassliebe.txt}


\newpage
\section{Quellcode}
\lstinputlisting[label=lst:girl, caption=Umsetzung des Girl-Headers, firstline=8, lastline=16]{../Umsetzung/girl.hpp}
\lstinputlisting[label=lst:main, caption=Umsetzung des Programmes, firstline=24]{../Umsetzung/main.cpp}


\nocite{*} % Alles REIN
\printbibliography[heading=bibintoc]%
\end{document}
