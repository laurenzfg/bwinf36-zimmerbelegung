\documentclass[a4paper,10pt,ngerman]{scrartcl}
\usepackage[ngerman]{babel}

\usepackage[T1]{fontenc}
\usepackage[utf8]{inputenc}
\usepackage[a4paper,margin=2.5cm]{geometry}

% Die nächsten drei Felder bitte anpassen:
\newcommand{\Name}{Laurenz Grote} % Teamname oder eigenen Namen angeben
\newcommand{\Einsendenummer}{Team 00905 / Teilnahme 44325}
\newcommand{\Aufgabe}{Aufgabe 1: Zimmerbelegung}

% Aliase zum abdrucken von Shell-CMDs
\newcommand{\shellcmd}[1]{\texttt{\$ #1}\\}
\newcommand{\shellout}[1]{\texttt{#1}\\}

% Kopf- und Fußzeilen
\usepackage{scrlayer-scrpage}
\setkomafont{pageheadfoot}{\textrm}
\ifoot{\Name}
\cfoot{\thepage}
\chead{\Aufgabe}
\ofoot{\Einsendenummer}

% Für mathematische Befehle und Symbole
\usepackage{amsmath}
\usepackage{amsthm}
\usepackage{amssymb}
\newtheorem{annahme}{Annahme}[section]

% Für Bilder
\usepackage{graphicx}

% Für Quelltext
\usepackage{listings}
\usepackage{xcolor}
\definecolor{mygreen}{rgb}{0,0.6,0}
\definecolor{mygray}{rgb}{0.5,0.5,0.5}
\definecolor{mymauve}{rgb}{0.58,0,0.82}
\lstset{language={C++},
  keywordstyle=\color{blue},commentstyle=\color{mygreen},
  stringstyle=\color{mymauve},rulecolor=\color{black},
  basicstyle=\footnotesize\ttfamily,numberstyle=\tiny\color{mygray},
  captionpos=t, % sets the caption-position to bottom
  keepspaces=true, % keeps spaces in text
  numbers=left, numbersep=5pt, showspaces=false,showstringspaces=true,
  showtabs=false, stepnumber=1, tabsize=2, title=\lstname,
  literate=%
    {Ö}{{\"O}}1
    {Ä}{{\"A}}1
    {Ü}{{\"U}}1
    {ß}{{\ss}}1
    {ü}{{\"u}}1
    {ä}{{\"a}}1
    {ö}{{\"o}}1
}
% Für die Bibliographie
\usepackage[babel,german=quotes]{csquotes}
\usepackage[citestyle=authortitle-icomp]{biblatex}
\addbibresource{literatur.bib}

% Diese beiden Pakete müssen als letztes geladen werden
\usepackage{hyperref} % Anklickbare Links im Dokument

% Daten für die Titelseite
\title{\Aufgabe}
\author{\Name / \Einsendenummer}
\date{27. November 2017}

\begin{document}
\maketitle
\tableofcontents

\section{Lösungsidee}
\subsection{Abstraktion}
Die Aufgabenstellung habe ich als Aufgabe aus dem Bereich der Graphentheorie interpretiert.

Im Kontext der Graphentheorie entsprechen Mädchen Knoten. Zimmerbelegungswünsche entsprechen folglich Kanten.
Nach Einlesen der Belegungswünsche entsteht somit ein gerichteter Graph. Falls alle Belegungswünsche auf Gegenseitigkeit beruhen, entsteht zu jeder Kante ein symmetrisch verlaufendes Gegenstück.
Der Graph ist in diesem Fall also  ungerichtet.
Da für meine im folgenden aufgeführte Lösungsidee jedoch in jedem Fall ein ungerichteter Graph erforderlich ist, muss für jede Kante eine entgegengesetzt verlaufende parallele Kante hinzugefügt werden. Somit ist der Graph über die Zimmerbelegungswünsche in jedem Fall ungerichtet.

Für die Verwaltung der "`negativen Wünsche"', also den Mädchen, mit denen ein Mädchen auf keinen Fall in einem Zimmer sein möchte, könnte ein ähnlicher Graph verwendet werden.
Da dies für den im folgenden aufgeführten Algorithmus nicht nötig ist, reicht es, die negativen Paarungen als Liste zu verwalten.

\subsection{Ermittlung der Zimmerbelegung}
Zusammenhängigkeitskomponenten (TODO definieren) im resultierenden ungerichteten Graphen entsprechen der Zimmerbelegung unter Berücksichtigung aller Wünsche.
Dies ist der Fall, da sowohl ein einseitiger als auch ein zweiseitiger Zimmerwunsch zur Folge hat, das beide Mädchen sich ein Zimmer teilen müssen.
Außerdem muss Mädchen A, welches sich ein Zimmer mit Mädchen B wünscht, auch in einem Zimmer mit Mädchen C, welches sich ebenfalls ein Zimmer mit Mädchen B wünscht, sein. Der Belegungswünsche wirken sich also im Graphen transitiv aus.

Die Menge der in einer Zusammenhangskomponente mit der fortlaufenden Nummer \(n\) enthaltenen Knoten entspricht also dem Zimmer \(n\), welches mit dem Mädchen in der Menge belegt ist.

\subsection{Überprüfung der Validität der Zimmerbelegung}
Um zu überprüfen, ob bei der Berücksichtichtigung aller Wünsche Zimmer so belegt wurden, dass keine negativen Wünsche verletzt wurden reicht es, zu überprüfen, ob beide Personen eines negativen Wunsches im gleichen Zimmer (Zusammenhangskomponente) platziert wurden.
Wenn dies bei keinem negativen Wunsch der Fall ist, ist die Berücksichtigung aller Wünsche möglich.
Schließlich sind alle Mädchen, zwischen denen ein negativer Wunsch in beliebige Richtung existiert, in verschiedenen Zimmern untergebracht. Dann können schlussendlich alle Zimmer ausgegeben werden.

Ansonsten können nicht alle Belegungswünsche erfüllt werden.


\section{Umsetzung}
\subsection{Parsing}

\subsection{Ermittlung der Zusammenhangskomponenten}
Da für die oben aufgeführte Lösungsidee die genaue Struktur des Graphens von keiner Relevanz ist, ist eine traditionelle Abbildung des Graphens mittels einer Adjazenzliste oder einer Adjazenzmatrix nicht nötig.
Da nur die Information der Zugehörigkeit zu einer Zusammenhangskomponente relevant ist, bietet sich eine Union-Find-Datenstruktur\autocite[S. 238]{Sedgewick2014} an. Diese ordnet jedem Knoten eine Zusammenhangskomponente zu. Die genauen Kanten werden hingegen nicht gespeichert.

Da eine typische Klasse eine größe von 30 Mädchen nicht überschreitet, habe ich unter den in der Literaturquelle vorgestellten Implementationen "`Quick-Find"'\autocite[S. 245]{Sedgewick2014} für ausreichend effizient befunden.

In meiner Implementation sind alle Knoten (Mädchen) Schlüssel einer Map. Als Wert haben sie eine Person in deren Zimmer sie sind. Der Wert ist hierbei ein Zeiger auf die Mapposition des jeweiligen Mädchens. Wichtig ist hierbei, dass es für jedes Zimmer das gleiche Mädchen als Identifier dient.



\section{Beispiele}
Falls Sie eigene Beispiele testen möchten,
rufen Sie das Programm bitte folgendermaßen auf:

\shellcmd{./zimmerbelegung.x dateiname}

Getestet habe ich meinen Quellcode unter gcc 7.1.1-3 auf einem 64bit-Fedora.
Mit folgendem Kommando können Sie mein Programm selbst kompilieren:

\shellcmd{g++ -O -o zimmerbelegung.x main.cpp girl.cpp}

Hier finden Sie die Ausgaben zu allen Beispielen der BwInf-Website:

\lstinputlisting[breaklines=true]{../Beispiele/Ausgaben/z1.txt}
\lstinputlisting[breaklines=true]{../Beispiele/Ausgaben/z2.txt}
\lstinputlisting[breaklines=true]{../Beispiele/Ausgaben/z3.txt}
\lstinputlisting[breaklines=true]{../Beispiele/Ausgaben/z4.txt}
\lstinputlisting[breaklines=true]{../Beispiele/Ausgaben/z5.txt}
\lstinputlisting[breaklines=true]{../Beispiele/Ausgaben/z6.txt}

Zur weiteren Überprüfung meines Programmes habe ich einige weitere Testfälle konstruiert.

In \texttt{keineWuensche.txt} ist allen die Zimmerbelegung völlig egal,
sie haben weder Wünsche noch Konflikte.
Folglich werden alle in Einzelzimmern platziert.
\lstinputlisting[breaklines=true]{../Beispiele/Ausgaben/keineWuensche.txt}

In \texttt{everybodysDarling.txt} möchten alle mit "`Luisa"' wohnen,
diese hat weder Wünsche noch Konflikte.
Folglich werden alle in einem gemeinsamen Zimmer platziert.
\lstinputlisting[breaklines=true]{../Beispiele/Ausgaben/everybodysDarling.txt}

In \texttt{hassliebe.txt} existiert ein Widerspruch, "`Luca"' gibt "`Julius"'
gleichzeitig als Wunsch als auch als Konflikt an.
Folglich wird "`Impossible"' ausgegeben.
\lstinputlisting[breaklines=true]{../Beispiele/Ausgaben/hassliebe.txt}


\newpage
\section{Quellcode}
\lstinputlisting[label=lst:girl, caption=Umsetzung des Girl-Headers, firstline=8, lastline=16]{../Umsetzung/girl.hpp}
\lstinputlisting[label=lst:main, caption=Umsetzung des Programmes, firstline=24]{../Umsetzung/main.cpp}


\nocite{*} % Alles REIN
\printbibliography[heading=bibintoc]%
\end{document}
