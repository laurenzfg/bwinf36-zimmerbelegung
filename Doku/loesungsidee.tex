\subsection{Abstraktion}
Die Aufgabenstellung habe ich als Aufgabe aus dem Bereich der Graphentheorie interpretiert.

Im Kontext der Graphentheorie entsprechen Mädchen Knoten. Zimmerbelegungswünsche entsprechen folglich Kanten.
Nach Einlesen der Belegungswünsche entsteht somit ein gerichteter Graph. Falls alle Belegungswünsche auf Gegenseitigkeit beruhen, entsteht zu jeder Kante ein symmetrisch verlaufendes Gegenstück.
Der Graph ist in diesem Fall also  ungerichtet.
Da für meine im folgenden aufgeführte Lösungsidee jedoch in jedem Fall ein ungerichteter Graph erforderlich ist, muss für jede Kante eine entgegengesetzt verlaufende parallele Kante hinzugefügt werden. Somit ist der Graph über die Zimmerbelegungswünsche in jedem Fall ungerichtet.

Für die Verwaltung der "`negativen Wünsche"', also den Mädchen, mit denen ein Mädchen auf keinen Fall in einem Zimmer sein möchte, könnte ein ähnlicher Graph verwendet werden.
Da dies für den im folgenden aufgeführten Algorithmus nicht nötig ist, reicht es, die negativen Paarungen als Liste zu verwalten.

\subsection{Ermittlung der Zimmerbelegung}
Zusammenhängigkeitskomponenten (TODO definieren) im resultierenden ungerichteten Graphen entsprechen der Zimmerbelegung unter Berücksichtigung aller Wünsche.
Dies ist der Fall, da sowohl ein einseitiger als auch ein zweiseitiger Zimmerwunsch zur Folge hat, das beide Mädchen sich ein Zimmer teilen müssen.
Außerdem muss Mädchen A, welches sich ein Zimmer mit Mädchen B wünscht, auch in einem Zimmer mit Mädchen C, welches sich ebenfalls ein Zimmer mit Mädchen B wünscht, sein. Der Belegungswünsche wirken sich also im Graphen transitiv aus.

Die Menge der in einer Zusammenhangskomponente mit der fortlaufenden Nummer \(n\) enthaltenen Knoten entspricht also dem Zimmer \(n\), welches mit dem Mädchen in der Menge belegt ist.

\subsection{Überprüfung der Validität der Zimmerbelegung}
Um zu überprüfen, ob bei der Berücksichtichtigung aller Wünsche Zimmer so belegt wurden, dass keine negativen Wünsche verletzt wurden reicht es, zu überprüfen, ob beide Personen eines negativen Wunsches im gleichen Zimmer (Zusammenhangskomponente) platziert wurden.
Wenn dies bei keinem negativen Wunsch der Fall ist, ist die Berücksichtigung aller Wünsche möglich.
Schließlich sind alle Mädchen, zwischen denen ein negativer Wunsch in beliebige Richtung existiert, in verschiedenen Zimmern untergebracht. Dann können schlussendlich alle Zimmer ausgegeben werden.

Ansonsten können nicht alle Belegungswünsche erfüllt werden.
