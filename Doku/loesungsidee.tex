\subsection{Abstraktion}
Die Aufgabenstellung habe ich als Aufgabe aus dem Bereich der Graphentheorie
interpretiert.

Im Kontext der Graphentheorie entsprechen Mädchen Knoten.
Zimmerbelegungswünsche entsprechen folglich Kanten.
Im entstehenden Graphen zeigt dann beispielweise im ersten Beispiel eine Kante von Anna
zu Paula, da Anna sich ein Zimmer mit dieser wünscht.
Nach Einlesen der Belegungswünsche entsteht somit ein gerichteter Graph.

Die möglichen Konflikte, also die Mädchen, mit denen ein Mädchen auf keinen Fall
in einem Zimmer sein möchte, können ebenso in einem gerichteter Graphen verwaltet werden.
Hier entspricht eine Kante dann einer augeschlossenen gemeinsamen Platzierung der beiden Knoten der Kante.

\subsection{Ermittlung der Zimmerbelegung}
\begin{annahme} \label{theo:annahme}
Schwache Zusammenhängigkeitskomponenten im Belegungswunschgraphen entsprechen den Zimmerbelegungen unter Berücksichtigung aller Wünsche.
Schwache Zusammenhangskomponenten\autocite[Definiton. Gerichtete Graphen]{WikiZus}
entsprechen den Teilgraphen des Graphens, in  denen zwischen jedem Knoten eine Verbindung über einen beliebig langen Pfad bestünde,
ersetzte man alle gerichteten Kanten durch entsprechende ungerichtete.
\end{annahme}

Diese Annahme ist korrekt, da sowohl ein einseitiger als auch ein beidseitiger Zimmerwunsch zur Folge hat, dass beide Mädchen sich ein Zimmer teilen müssen. Daher können die gerichten Kanten wie ungerichtete Kanten behandelt werden.

Außerdem muss Mädchen A, welches sich ein Zimmer mit Mädchen B wünscht, auch in einem Zimmer mit Mädchen C, welches sich ebenfalls ein Zimmer mit Mädchen B wünscht, sein. Die Belegungswünsche wirken sich also im Graphen \textit{transitiv} aus. Daher kann der Pfad, über den sich die Notwendigkeit eines gemeinsamen Zimmers ergibt, beliebig lang sein.

\subsection{Überprüfung der Validität der Zimmerbelegung}
Um zu überprüfen, ob die nach Annahme \ref{theo:annahme} berechnete Zimmerverteilung im Sinne der Aufgabenstellung gültig ist, reicht es,
für jede Konfliktkante zu überprüfen, ob beide Knoten der Kante im gleichen Zimmer sind.
Wenn dies für keine Konfliktkante der Fall ist, befinden sich alle Mädchen, zwischen denen ein Konflikt besteht, in verscheidenen Zimmern.

Zur Ausgabe der errechneten Zimmerbelegungen müssen alle Mädchen innerhalb einer Zusammenhangskomponente ausgegeben werden.
