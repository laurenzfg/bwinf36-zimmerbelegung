\subsection{Parsing und Datenstrukturen}
Das auf der Wettbewerbsseite vorgestellte Eingabeformat legt die Repräsentation
beider Graphen mithilfe von Adjazenzlisten nahe.
Schließlich sind in der Eingabedatei zu jedem Knoten alle verbundenen Knoten notiert.

Adjazenzlisten sind eine von drei Möglichkeiten, einen Graphen zu speichern. Hierbei wird 
für jeden Knoten in einer Liste notiert, zu welchen Knoten eine Verbindung (=Kante)
besteht.

Meine Implementation speichert die beiden Adjazenzlisten (Wünsche sowie Konflikte)
als Vektor, die Zimmernummer und den Namen eines jeden Mädchens gebündelt in einem
Aggregat\footnote{In der Programmiersprache C++ ist ein Aggregat eine Möglichkeit der
Sprache, verschiedene Variablen in einem Objekt zu bündeln. Vorteil ist, dass der
Konstruktor implizit und der Zugriff auf die Felder denkbar einfach ist.}.
Alle Aggregate werden in einer Map gespeichert, bei der der Name des Mädchens den 
Schlüssel bildet.

So können alle Daten zu einem Mädchen schnell abgerufen werden.
Die Zimmernummer wird zunächst leer initialisiert.

\subsection{Ermittlung der Zusammenhangskomponenten}
Zur Ermittlung der Zimmer nach Annahme \ref{theo:annahme} müssen die schwachen
Zusammenhangskomponenten des Graphens ermittelt werden. 
Da nur die Information der Zugehörigkeit zu einer Zusammenhangskomponente relevant ist,
bietet sich eine Union-Find-Datenstruktur\autocite[S. 238]{Sedgewick2014} an.
Diese ordnet nach einmaliger Berechnung jedem Knoten eine
Zusammenhangskomponente (=Zimmer) zu.

In einer Union-Find-Datenstruktur werden nach und nach alle Kanten eingegeben
(Operation \textit{union}). Dabei werden die Zusammenhangskomponenten ermittelt, die
mit der Operation \textit{find} abgerufen werden können.

Da nur einmal die Zusammenhangskomponenten berchnet werden, aber im folgenden
Programmablauf oft Zusammenhangskomponeten verglichen werden, 
ist meine Implementation an die in der Literaturquelle vorgestellte Implementation 
"`Quick-Find"'\autocite[S. 245]{Sedgewick2014} angelehnt.
Ein weiter Vorteil dieser Strategie ist, dass der Graph zumindest bei meiner
im folgenden skizzierten Implementation nicht erst in einen ungerichteten Graphen
umgeformt werden muss.
Es reicht, eine Kante in eine Richtung einmal einzugeben.

Zunächst wird jedem Mädchen eine fortlaufende Zimmer-ID zugeordnet.
Im Kontext der Aufgabenstellung wird also jedem Mädchen ein Einzelzimmer zugeordnet.
Danach können die Belegungswünsche beachtet werden.

Dafür wird bei jedem Mädchen über alle Belegungswünsche iteriert.
Wenn nun die beiden Mädchen eines Belegungswunsches (=Kante im Wunschgraphen)
nicht im gleichen Zimmer sind, müssen alle Mädchen beider betroffenen Zimmer in das
gleiche Zimmer wechseln.
In welches Zimmer gewechselt wird ist irrelevant.
In meiner Implementierung wird in das Zimmer, von dem die Kante ausgeht, gewechselt.
Der Wechselvorgang ist so implementiert, dass die Zimmer-ID aller Mädchen in dem einem
Zimmer auf die ID des anderen Zimmers gesetzt wird. So werden beide Zimmer
zusammengeschlossen. Dies entspricht der \textit{union}-Operation.

Wichtig ist, dass alle Mädchen das Zimmer wechseln. Nicht nur die Mädchen,
die direkt mit einer Kante verbunden sind.
Schließlich ist, wie in der Lösungsidee aufgeführt, Transitivität gegeben.
Durch den Umzug aller Mädchen eines Zimmers wird gewährleistet,
dass kein schon berücksichtiger Wunsch verletzt wird.

Nach Betrachtung aller Kanten haben alle Mädchen,
die zur Berücksichtigung aller Wünsche in einem Zimmer sein müssen,
die gleiche Zimmer-ID.

Die \textit{find}-Operation entspricht also dem Auslesen der Zimmernummer aus dem Aggregat des jeweiligen Mädchens.

\subsection{Überprüfung der Gültigkeit der Zimmerbelegung}
Um zu überprüfen, ob zwei Mädchen, zwischen denen ein Konflikt besteht,
in einem gemeinsamen Zimmer sind,
wird wiederum über alle Konfliktkanten eines jeden Mädchens iteriert.
Sollten zwei Mädchen einer Konfliktkante die gleiche Zimmer-ID haben, ist keine Belegung möglich,
bei der sowohl alle Wünsche als auch alle Konflikte berücksichtigt werden.
Schließlich befinden sich zwei Mädchen, zwischen denen ein Konflikt besteht, im gleichen
Zimmer.

Sollten alle Mädchen, zwischen denen ein Konflikt besteht,
in jeweils verschiedenen Zimmern untergebracht sein,
ist eine gültige Zimmerbelegung möglich.

\subsection{Ausgabe}
Dann werden alle Zimmerbelegungen ausgegeben.
Dies geschieht, indem für jede mögliche Zimmer-ID, also \(0 .. n-1\),
alle Mädchen, die diese ID haben, ausgegeben werden.
Viele Zimmer sind aufgrund der Zusammenlegungen leer,
diese werden dann entsprechend übersprungen.

Falls keine gültige Zimmerbelegung möglich ist, erfolgt die Ausgabe "`Impossible"'.
